\documentclass[letterpaper,10pt]{article}

\usepackage{graphicx}                                        
\usepackage{amssymb}                                         
\usepackage{amsmath}                                         
\usepackage{amsthm}                                          

\usepackage{alltt}                                           
\usepackage{float}
\usepackage{color}
\usepackage{url}

\usepackage{balance}
\usepackage[TABBOTCAP, tight]{subfigure}
\usepackage{enumitem}
\usepackage{pstricks, pst-node}

\usepackage{geometry}
\geometry{textheight=8.5in, textwidth=6in}

\newcommand{\cred}[1]{{\color{red}#1}}
\newcommand{\cblue}[1]{{\color{blue}#1}}

\usepackage{hyperref}

\def\name{Dylan Camus}

%pull in the necessary preamble matter for pygments output
\input{pygments.tex}

% The following metadata will show up in the PDF properties
\hypersetup{
   colorlinks = false,
   urlcolor = black,
   pdfauthor = {\name},
   pdfkeywords = {cs444 ``operating systems'' summary},
   pdftitle = {CS 444 Weekly Summary: Week Five},
   pdfsubject = {CS 444 Weekly Summary: Week Five},
   pdfpagemode = UseNone
}

\parindent = 0.0 in
\parskip = 0.1 in

\title{CS444: Bottom Halves and Memory Management}
\author{Dylan Camus}
\date{\today}
\begin{document}
\maketitle
Robert Love's book, "Linux Kernel Development(2010)", chapter 14, asserts that bottom halves and memory management in the Linux kernel are fundemental concepts to understanding how to program within the kernel. Love supports this claim by explaining how softirqs, tasklets, and work queues are implemented in regards to interrupts and synchonization and how memory is allocated in the kernel. Love's purpose is to give the reader a sense of why bottom halves are used and how they affect syncronization, as well as how memory can sometimes be difficult to obtain from the kernel and the differences between kernel and user-space development. Considering the technical language used in this chapter, Love is writing to an audience who already has an understanding of the basics of operating systems and is familiar with the C programming language.
\end{document}
