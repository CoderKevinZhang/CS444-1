\documentclass[letterpaper,10pt]{article}

\usepackage{graphicx}                                        
\usepackage{amssymb}                                         
\usepackage{amsmath}                                         
\usepackage{amsthm}                                          

\usepackage{alltt}                                           
\usepackage{float}
\usepackage{color}
\usepackage{url}

\usepackage{balance}
\usepackage[TABBOTCAP, tight]{subfigure}
\usepackage{enumitem}
\usepackage{pstricks, pst-node}

\usepackage{geometry}
\geometry{textheight=8.5in, textwidth=6in}

\newcommand{\cred}[1]{{\color{red}#1}}
\newcommand{\cblue}[1]{{\color{blue}#1}}

\usepackage{hyperref}

\def\name{Dylan Camus}

%pull in the necessary preamble matter for pygments output
\input{pygments.tex}

% The following metadata will show up in the PDF properties
\hypersetup{
   colorlinks = false,
   urlcolor = black,
   pdfauthor = {\name},
   pdfkeywords = {cs444 ``operating systems'' summary},
   pdftitle = {CS 444 Weekly Summary: Week Ten},
   pdfsubject = {CS 444 Weekly Summary: Week Ten},
   pdfpagemode = UseNone
}

\parindent = 0.0 in
\parskip = 0.1 in

\title{CS444: Kernel Synchronization Methods and The Virtual Filesystem}
\author{Dylan Camus}
\date{\today}
\begin{document}
\maketitle
Robert Love's book, "Linux Kernel Development(2010)", chapters 10 and 13, asserts that synchronization and the virtual filesystem in the Linux kernel are fundemental concepts to understanding how to program within the kernel. Love supports this claim by explaining how the kernel enforces synchronization and concurrency with atomic operations, spin locks, and semaphores, and how the kernel implements the virtual filesystem to make it easier to implement new filesystems in linux. Love's purpose is to give the reader a sense of how the kernel handles the prevention of race conditions by ensuring correct synchronization and how the virtual filesystem and its various data structures and superblock is handled within the kernel. Considering the technical language used in this chapter, Love is writing to an audience who already has an understanding of the basics of operating systems and is familiar with the C programming language.
\end{document}
