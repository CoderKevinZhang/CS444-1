\documentclass[journal,letterpaper,draftclsnofoot,onecolumn,10pt]{IEEEtran}

\usepackage{graphicx}                                        
\usepackage{amssymb}                                         
\usepackage{amsmath}                                         
\usepackage{amsthm}                                          

\usepackage{alltt}                                           
\usepackage{float}
\usepackage{color}
\usepackage{url}

\usepackage{balance}
\usepackage[TABBOTCAP, tight]{subfigure}
\usepackage{enumitem}
\usepackage{pstricks, pst-node}

\usepackage{geometry}
\geometry{textheight=8.5in, textwidth=6in}

\newcommand{\cred}[1]{{\color{red}#1}}
\newcommand{\cblue}[1]{{\color{blue}#1}}

\usepackage{hyperref}

\def\name{Dylan Camus}

%pull in the necessary preamble matter for pygments output
\input{pygments.tex}

% The following metadata will show up in the PDF properties
\hypersetup{
   colorlinks = false,
   urlcolor = black,
   pdfauthor = {\name},
   pdfkeywords = {cs444 ``operating systems'' ``project one''},
   pdftitle = {CS 444 Project 1 Writeup},
   pdfsubject = {CS 444 Project 1 Writeup},
   pdfpagemode = UseNone
}

\parindent = 0.0 in
\parskip = 0.1 in

\begin{document}
\begin{titlepage}

\newcommand{\HRule}{\rule{\linewidth}{0.5mm}} % Defines a new command for the horizontal lines, change thickness here

\center % Center everything on the page
 
%---------------------------------------------------------------------------
%	HEADING SECTIONS
%---------------------------------------------------------------------------

\textsc{\LARGE Oregon State University}\\[1.5cm] % Name of your university/college
\textsc{\Large Operating Systems II}\\[0.5cm] % Major heading such as course name
\textsc{\large CS444}\\[0.5cm] % Minor heading such as course title

%---------------------------------------------------------------------------
%	TITLE SECTION
%---------------------------------------------------------------------------

\HRule \\[0.4cm]
{ \huge \bfseries Project 3 Writeup}\\[0.4cm] % Title of your document
\HRule \\[1.5cm]
 
%---------------------------------------------------------------------------
%	AUTHOR SECTION
%---------------------------------------------------------------------------

\begin{minipage}{0.4\textwidth}
   \begin{flushleft} \large
      \emph{Author:}\\
      Dylan \textsc{Camus} % Your name
   \end{flushleft}
\end{minipage}
~
\begin{minipage}{0.4\textwidth}
   \begin{flushright} \large
      \emph{Professor:} \\
      Prof. Kevin \textsc{McGrath} % Supervisor's Name
   \end{flushright}
\end{minipage}\\[4cm]

% If you don't want a supervisor, uncomment the two lines below and remove the section above
%\Large \emph{Author:}\\
%John \textsc{Smith}\\[3cm] % Your name

%---------------------------------------------------------------------------
%	DATE SECTION
%---------------------------------------------------------------------------

{\large \today}\\[3cm] % Date, change the \today to a set date if you want to be precise

%---------------------------------------------------------------------------
%	ABSTRACT SECTION
%---------------------------------------------------------------------------
\begin{abstract}
This project involved the use of a block device driver. Specifically, a block device driver was developed that acted as a RAMdisk. This device would read and write data. A filesystem was mounted on the device. The device was encrypted using Linux crypto.
\end{abstract}

\vfill % Fill the rest of the page with whitespace

\pagebreak

\end{titlepage}

\section{Design}
The structure of the rdcrypto block device was very similar to the sbd block device that Kevin told us to use as a base. The only function that was changed was the transfer function. This is the function that handles reads and writes. The changes that I made where to add encryption when writing to the device and to add decryption when reading from the device. Otherwise, the device is essentially the same as the sbd device; it stores data.
\section{Questions}
\subsection{What do you think the main point of this assignment is?}
I think the main point of the assignment was to become familiar with the idea of patching the Linux kernel and to take a look at how modules are loaded into the kernel. Also, it obviously was meant to teach us about block devices and Linux crypto.
\subsection{How did you personally approach the problem? Design decisions, algorithm, etc.}
I approached the problem by researching how to use the Linux crypto library. Once I had a solid base of knowledge, there were only really 20 - 30 lines of code that needed to be written. The main problem I had was initializing the crypto. Using the crypto was just two functions or so, so it was relativily easy.
\subsection{How did you ensure your solution was correct? Testing details, for instance.}
I ensured my solution was correct by using print statements to show when my device was reading or writing, and to show the data before and after encryption/decryption.
{What did you learn?}
I learned how to make Linux patch files, how to configure a module in Linux, and how to write a Linux block device with encryption.

\section{Work Log}

\begin{figure}[H]
   \begin{tabular}{c | c | c}
      Date & Hours & Detail\\
      \hline
      5/15 & 4 & I researched the Linux crypto library and worked on the code\\
      \hline
      5/16 & 10 & I struggled to get the module to copy to the VM\\
      \hline
   \end{tabular}
   \caption{Dates and number of hours spent working on project 2}
\end{figure}
      
\end{document}
